%%%%%%%%%%%%%%%%%%%%%%%%%%%%%%%%%%%%%%%%%%%%%%%%%%%%%%%%%%%%%%%%%%%%%%%%
%
%.IDENTIFICATION proposal-template.tex,v 2                       20/01/2019
%.LANGUAGE       LaTeX
%.PURPOSE        LaTeX template form for observing field trip proposal
%.AUTHOR         Antonio Martin-Carrillo (UCD)
% 
%
%                        4 PAGES MAXIMUM
%%%%%%%%%%%%%%%%%%%%%%%%%%%%%%%%%%%%%%%%%%%%%%%%%%%%%%%%%%%%%%%%%%%%%%%%
\documentclass[11pt]{article}     
\usepackage{pdfpages}
\usepackage{multicol}
%
%
% DO NOT CHANGE ANYTHING IN THE FOLLOWING 6 LINES! (but see warning 1.)
\textheight=247mm
\textwidth=180mm
\topmargin=-7mm
\oddsidemargin=-10mm
\evensidemargin=-10mm
\parindent 10pt
%
\begin{document}
%
%
%---- ENTRY 1 ----------------------------------------------------------
%
\centerline{\large{\bf
{Title of your proposal}}}  

\medskip
%---- ENTRY 2 ----------------------------------------------------------
% 
\centerline{\bf PI: 
{A. Healy}}
 
\bigskip
%---- ENTRY 3 ----------------------------------------------------------
%
\noindent {\bf 1. Abstract}
\smallskip\\
\textbf{MAX length of the proposal 4 PAGES - including references}

Concise abstract of your proposal to be given here. 
Brief summary of your science case and your total exposure time, filters to be used.

The main goal of studying galactic evolution is to understand how galaxies produce their stars over time.
The easiest way to study this is to measure star formation rates and stellar masses as a function of redshift
for a sample of the galaxies.
From photometric measurements of galaxies, the physical properties of resolved and unresolved galaxies can be
inferred from fitting the measurement to physical models for spectral energy distributions. The morphology of
galaxies can also be quantified by looking at their isophotes and radial profiles. Combining these approaches
to study galaxies of different ages allows for galactic evolution to be studied. This project aims to classify
galaxies, investigate their structure, and measure their physical properties in order to compare galaxies of the
same classification at different ages. This will be achieved through spectral energy distribution fitting and
constructing the radial profiles of the galaxies from photometric data collected using a 1.2-meter telescope

%%% Talk about exposure time and filters used

%---- ENTRY 4 ----------------------------------------------------------
%
% This is the place where the proposed programme is to be described.
% This description is composed of two different sections.
%
% A) Scientific rationale: scientific background of the project,
% pertinent references; justification for present
% proposal.
%
% B) Immediate objective of the proposal: what specifically is to 
% be observed and what can be learnt from the observations so that 
% the feasibility becomes clear.
%

\smallskip
\noindent {\bf 2. Description of the proposed programme\\}
\noindent {\sl A) Scientific Rationale:}
\smallskip\\
State the scientific background of the project, pertinent references; justification for the present proposal.

The purpose of this project is to study the different properties and structures of galaxies at various ages, or
redshift. This will be achieved by studying resolvable galaxies and unresolved galaxies which are candidates to
be the same morphological type. The main goal of studying galactic evolution is to understand how galaxies
produce their stars over time. The easiest way to study this is to measure star formation rates and stellar
masses as a function of redshift for a sample of the galaxies (Leja et al. 2017). To study the properties of the
galaxies spectral energy distribution fitting will be utilised. The morphology of the galaxies will be studied by
constructing isophotes and radial profiles from the photometric data collected.

%-----------------------------Figure Start------------------------------
\begin{figure}[h]
\begin{center}
%\hbox{
%\includegraphics[width=0.9\textwidth]{file.pdf}
%
% or in the case of two figures side by side
%\includegraphics[width=0.45\textwidth]{file1.pdf}
%\includegraphics[width=0.45\textwidth]{file2.pdf}
\end{center}
\caption{\footnotesize{
%%{{\it Left panel}: here you see.
%%{\it Right panel}: there you see
}}
\label{somethhihng}
\end{figure}
%-----------------------------Figure End--------------------------------

\smallskip
\noindent {\sl B) Immediate Objective:}
\smallskip\\
State what specifically is to be observed and what can be learnt 
from the observations, such that the feasibility becomes clear.

%
%---- ENTRY 5 ----------------------------------------------------------
%
% Provide a careful justification of the requested observing time,
% a feasibility study (expected signal-to-noise estimate 
% for the time requested; background estimates especially for extended 
% sources) and give information on the target visibility 
% 
% If proposing more than one target, indicate the priority. 
%
\smallskip
\noindent {\bf 3. Justification of requested observing time, 
feasibility and visibility}
\smallskip\\
Provide a careful justification of the requested observing time, a feasibility study (expected signal-to-noise estimate for the time requested...) and give information on the target visibility.
If asking for more than one target, indicate the priority.
Please do provide a list of back-up targets with coordinates, observing windows, time of transits (in the case of exoplanets) and exposures to use for each target. \textbf{You are allowed to have a table with backup targets in a 5th page}.

%
%---- ENTRY 6 ----------------------------------------------------------
% 
%
\smallskip
\noindent {\bf 4. Previous/complementary data}
\smallskip\\
%Add here information about other data available to you that is relevant to the project and that it will be used in the final report. 
%Demonstrate that during the first semester and beginning of this semester, you have build the skills required to successfully complete the proposed project.

Add here information about other data available to you that is relevant to the project and that it will be used in the final report. Mention what is the format of this data (are they magnitudes taken from a paper/database? are they images you need to analyse from the beginning? Include here how this data enhance your observations and how you will use it.

Archival images will also be used from the SDSS to provide data in the u, r, i, and z bands. These images can
be retrieved using the astroquery.sdss module from the SDSS database. These images are already reduced and contain information on the magnitudes of the objects as well as the WCS data
in their FITS file, so they do not need to be processed.


%
%---- ENTRY 7 ----------------------------------------------------------
%
% References
%
% Each reference should be separated by the LaTex command \smallskip\\
%
\smallskip
\noindent {\bf 5. References}
\smallskip

\noindent Name1 A., Name2 B., 2015, ApJ, 599, 111: Title of article1
\smallskip\\
Leja, Joel et al. 2017, Astrophysical Journal 837.2, p. 170: “Deriving physical properties from broadband photometry with Prospector: descrip-
tion of the model and a demonstration of its accuracy using 129 galaxies in the local Universe”. In: The
\smallskip\\



%%%%%%%%%%%%%%%%%%%%%%%%%%%%%%%%%%%%%%%%%%%%%%%%%%%%%%%%%%%%%%%%%%%%%%%%
%%%%% THE END %%%%%%%%%%%%%%%%%%%%%%%%%%%%%%%%%%%%%%%%%%%%%%%%%%%%%%%%%%
%%%%%%%%%%%%%%%%%%%%%%%%%%%%%%%%%%%%%%%%%%%%%%%%%%%%%%%%%%%%%%%%%%%%%%%%
\end{document}
%
% Everything following \end{document} is discarded. 
%%%%%%%%%%%%%%%%%%%%%%%%%%%%%%%%%%%%%%%%%%%%%%%%%%%%%%%%%%%%%%%%%%%%%%%%